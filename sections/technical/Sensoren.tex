\section{Sensoren}


\subsection{Messen der Lufttemperatur}
\subsection{Ermittlung der Niederschlagsmenge}
Dieses Unterkapitel befasst sich mit der Realisierung der Niederschlagsmessung. Diese soll nach einem Kipplöffelprinzip funktionieren und gemäss definierten Zielen eine Genauigkeit von $\pm$100 ml/$m^2$ aufweisen. Ausserdem soll als alternative zusätzlich ein Messbecher an der Wetterstation installiert werden, damit der Bauer die Niederschlagsmenge anhand einer Skala ablesen kann. In einem ersten Schritt soll das Kipplöffelprinzip näher erläutert werden. Darauf folgend sollen die Realisierung dieses Kipplöffels und anschliessend die Implementation in der Software thematisiert werden. Zu guter Letzt soll die Validierung des Teilsystems folgen.
\subsubsection*{Das Kipplöffelprinzip}
Das Prinzip des Kipplöffels wird in Abbildung GRAPHKIPP graphisch dargestellt.

GRAPHKIPP

GRAPHKIPP zeigt das Prinzip des Kipplöffels. Der Kipplöffel besteht im Grunde aus zwei Löffeln und ist in der Mitte mit dem Gehäuse befestigt. Regenwasser wird über eine Öffnung im Gehäusedeckel zum Kipplöffel befördert. Ist der Löffel mit Regenwasser gefüllt, so kippt dieser aufgrund des Gewichts und leert das Wasser über eine Öffnung im Gehäuseboden aus. Durch die Kippung wird der andere Löffel in die Ausgangsposition bewegt und kann sich nun mit Wasser füllen. Mit der Hilfe von Reedkontakten und Magneten wird die Anzahl der Kippbewegungen gezählt. Die Niederschlagsmenge ergibt sich aus der Anzahl Kippbewegungen, multipliziert mit dem Volumen des Kipplöffels.
\subsubsection*{Die Realisierung des Kipplöffels}

\subsubsection*{Implentation in der Software}
\subsubsection*{Validierung der Niederschlagsmessung}
 
\subsection{Ermittlung der Windgeschwindigkeit}
\subsection{Zählung der Sonnenstunden}
